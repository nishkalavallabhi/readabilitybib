
\documentclass[a4paper,12pt]{report}
\usepackage{url}
\usepackage{acl}
\usepackage{times}
\usepackage[utf8]{inputenc}
\usepackage[colorlinks,linkcolor=black,citecolor=black,urlcolor=black]{hyperref}

\title{A bibliography of automatic readability assessment}
\author	{}
\date{\today}

\begin{document}

\section{Traditional Readability Formulae and their creation processes}

Most of the researchers who worked on early readability formulae also gave significant contributions to their field (typically Educational Psychology). So, I am also adding wikipedia pages where they exist, to give a perspective.

\begin{itemize}\itemsep1ex
\item \cite{Thorndike-21} - describe a process of vocabulary list creation for schools. Wikipedia page of Edward Thorndike: \url{https://en.wikipedia.org/wiki/Edward_Thorndike}

\item \cite{Lively.Pressey-23} proposed a method to estimate the vocabulary burden of text books. Sidney L.Pressey's wikipedia page: \url{https://en.wikipedia.org/wiki/Sidney_L._Pressey}

\item \cite{Vogel.Washburne-28} proposed "Winnetka Formula", which perhaps for the first time considered syntactic aspects of text and validated their formula against reading test scores.

\item \cite{Waples.Tyler-31} studied adult reading instead of children and looked at the specific issues involved there.
   \begin{itemize}
    \item Wikipedia page on Waples: \url{https://en.wikipedia.org/wiki/Douglas_Waples}
    \item Wikipedia page on Tyler: \url{https://en.wikipedia.org/wiki/Ralph_W._Tyler}
   \end{itemize}

\item \cite{Patty.Painter-31} proposed another approach to measure the vocabulary burden in textbooks, as an improvement over previous approaches like \cite{Lively.Pressey-23}. 

\item \cite{Dale.Tyler-34} studied the factors influencing adult reading difficulty. Edgar Dale's Wikipedia page: \url{https://en.wikipedia.org/wiki/Edgar_Dale}

\item \cite{Gray.Leary-35} studied a wide range of factors that influence readability, with adult readers as their end-users. William Gray's wikipedia page: \url{https://en.wikipedia.org/wiki/William_S._Gray}

\item \cite{Lorge-39,Lorge-44,Thorndike.Lorge-44} worked on proposing simple, easy to use readability formulae to select reading passages for children.

\item \cite{Flesch-43,Flesch-48} studied the markers of readable style and proposed a readability formula which is still widely used. Wikipedia page on Rudolf Flesch: \url{https://en.wikipedia.org/wiki/Rudolf_Flesch}

\item \cite{Dale.Chall-48,Dale.Chall-48a} also proposed another readability formula, and continued improving it. Wikipedia page on Chall: \url{https://en.wikipedia.org/wiki/Jeanne_Chall}

\item \cite{Farr.Jenkins.ea-51} describes a simplification of the Flesch reading ease formula.

\item \cite{Klare-52} studied the measures of readability in written evaluation and also continued to work on various aspects of textual readability for another 4,5 decades. Some of his articles/books on this topic are: \cite{Klare.Mabry.ea-55a,Klare.Mabry.ea-55b,Klare.Nichols.ea-57,Klare-63,Klare-69,Klare-74,Klare-00,Klare-00a}. Wikipedia page on George Klare: \url{https://en.wikipedia.org/wiki/George_R._Klare}

\item \cite{Kandel.Moles-58} proposed a readability formula for French, based on Flesch formula (the article referred here is written in French.)

\item \cite{Aukerman-65} wrote a report on the readability of secondary school literature textbooks.

\item \cite{Bormuth-66} proposed a new approach to readability, based on cloze-tests and studied a wide range of linguistic properties in the process. He also worked extensively on the topic. 

\item \cite{Smith.Senter-67} described the ARI readability formula.

\item \cite{Bjornsson-68} described the LIX readability formula for Swedish (This book is written in Swedish.) 

\item \cite{Gunning-68} described an early version of the Gunning Fog Index. 

\item \cite{McLaughlin-69} proposed the SMOG readability formula.

\item \cite{Jakobsen-71} described a Danish version for LIX. (This book is written in Danish.)

\item \cite{Caylor.Sticht.ea-73} described the FORCAST readability formula for Military communication use.

\item \cite{McLaughlin-74} compared some of the existing readability formulae.

\item \cite{Granowsky.Botel-74} described a new syntactic complexity formula.

\item \cite{Coleman.Liau-75} proposed another readability formula.

\item \cite{Kincaid.Fishburne.ea-75} described the creation of readability formulae for Navy personnel.

\end{itemize}


\section{Theoretical perspectives}

\section{User based experiments}

\section{corpora creation/description/analysis}

\begin{itemize}
\item \cite{Miller.Coleman-67} described a set of 36 prose passages calibrated for text complexity. 

\end{itemize}

\section{Related standardized tests and resources}

\begin{itemize}
\item \cite{McCall.Crabbs-26}

\item \cite{Taylor-53} - cloze test as a measure of readability.

\item General Service List \cite{West-53} and its enhanced versions. More details can be seen in the Wikipedia page of GSL: \url{https://en.wikipedia.org/wiki/General_Service_List}. 
\end{itemize}

\section{Feature engineering and Machine Learning}

\section{Evaluation}

\section{Applications of readability analysis}

\section{Various Theses}

\section{Readability related workshops, symposiums etc.}

\bibliographystyle{acl}
\bibliography{readability}
\end{document}
